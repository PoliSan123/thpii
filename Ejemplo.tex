
% Default to the notebook output style

    


% Inherit from the specified cell style.




    
\documentclass[11pt]{article}

    
    
    \usepackage[T1]{fontenc}
    % Nicer default font (+ math font) than Computer Modern for most use cases
    \usepackage{mathpazo}

    % Basic figure setup, for now with no caption control since it's done
    % automatically by Pandoc (which extracts ![](path) syntax from Markdown).
    \usepackage{graphicx}
    % We will generate all images so they have a width \maxwidth. This means
    % that they will get their normal width if they fit onto the page, but
    % are scaled down if they would overflow the margins.
    \makeatletter
    \def\maxwidth{\ifdim\Gin@nat@width>\linewidth\linewidth
    \else\Gin@nat@width\fi}
    \makeatother
    \let\Oldincludegraphics\includegraphics
    % Set max figure width to be 80% of text width, for now hardcoded.
    \renewcommand{\includegraphics}[1]{\Oldincludegraphics[width=.8\maxwidth]{#1}}
    % Ensure that by default, figures have no caption (until we provide a
    % proper Figure object with a Caption API and a way to capture that
    % in the conversion process - todo).
    \usepackage{caption}
    \DeclareCaptionLabelFormat{nolabel}{}
    \captionsetup{labelformat=nolabel}

    \usepackage{adjustbox} % Used to constrain images to a maximum size 
    \usepackage{xcolor} % Allow colors to be defined
    \usepackage{enumerate} % Needed for markdown enumerations to work
    \usepackage{geometry} % Used to adjust the document margins
    \usepackage{amsmath} % Equations
    \usepackage{amssymb} % Equations
    \usepackage{textcomp} % defines textquotesingle
    % Hack from http://tex.stackexchange.com/a/47451/13684:
    \AtBeginDocument{%
        \def\PYZsq{\textquotesingle}% Upright quotes in Pygmentized code
    }
    \usepackage{upquote} % Upright quotes for verbatim code
    \usepackage{eurosym} % defines \euro
    \usepackage[mathletters]{ucs} % Extended unicode (utf-8) support
    \usepackage[utf8x]{inputenc} % Allow utf-8 characters in the tex document
    \usepackage{fancyvrb} % verbatim replacement that allows latex
    \usepackage{grffile} % extends the file name processing of package graphics 
                         % to support a larger range 
    % The hyperref package gives us a pdf with properly built
    % internal navigation ('pdf bookmarks' for the table of contents,
    % internal cross-reference links, web links for URLs, etc.)
    \usepackage{hyperref}
    \usepackage{longtable} % longtable support required by pandoc >1.10
    \usepackage{booktabs}  % table support for pandoc > 1.12.2
    \usepackage[inline]{enumitem} % IRkernel/repr support (it uses the enumerate* environment)
    \usepackage[normalem]{ulem} % ulem is needed to support strikethroughs (\sout)
                                % normalem makes italics be italics, not underlines
    \usepackage{mathrsfs}
    

    
    
    % Colors for the hyperref package
    \definecolor{urlcolor}{rgb}{0,.145,.698}
    \definecolor{linkcolor}{rgb}{.71,0.21,0.01}
    \definecolor{citecolor}{rgb}{.12,.54,.11}

    % ANSI colors
    \definecolor{ansi-black}{HTML}{3E424D}
    \definecolor{ansi-black-intense}{HTML}{282C36}
    \definecolor{ansi-red}{HTML}{E75C58}
    \definecolor{ansi-red-intense}{HTML}{B22B31}
    \definecolor{ansi-green}{HTML}{00A250}
    \definecolor{ansi-green-intense}{HTML}{007427}
    \definecolor{ansi-yellow}{HTML}{DDB62B}
    \definecolor{ansi-yellow-intense}{HTML}{B27D12}
    \definecolor{ansi-blue}{HTML}{208FFB}
    \definecolor{ansi-blue-intense}{HTML}{0065CA}
    \definecolor{ansi-magenta}{HTML}{D160C4}
    \definecolor{ansi-magenta-intense}{HTML}{A03196}
    \definecolor{ansi-cyan}{HTML}{60C6C8}
    \definecolor{ansi-cyan-intense}{HTML}{258F8F}
    \definecolor{ansi-white}{HTML}{C5C1B4}
    \definecolor{ansi-white-intense}{HTML}{A1A6B2}
    \definecolor{ansi-default-inverse-fg}{HTML}{FFFFFF}
    \definecolor{ansi-default-inverse-bg}{HTML}{000000}

    % commands and environments needed by pandoc snippets
    % extracted from the output of `pandoc -s`
    \providecommand{\tightlist}{%
      \setlength{\itemsep}{0pt}\setlength{\parskip}{0pt}}
    \DefineVerbatimEnvironment{Highlighting}{Verbatim}{commandchars=\\\{\}}
    % Add ',fontsize=\small' for more characters per line
    \newenvironment{Shaded}{}{}
    \newcommand{\KeywordTok}[1]{\textcolor[rgb]{0.00,0.44,0.13}{\textbf{{#1}}}}
    \newcommand{\DataTypeTok}[1]{\textcolor[rgb]{0.56,0.13,0.00}{{#1}}}
    \newcommand{\DecValTok}[1]{\textcolor[rgb]{0.25,0.63,0.44}{{#1}}}
    \newcommand{\BaseNTok}[1]{\textcolor[rgb]{0.25,0.63,0.44}{{#1}}}
    \newcommand{\FloatTok}[1]{\textcolor[rgb]{0.25,0.63,0.44}{{#1}}}
    \newcommand{\CharTok}[1]{\textcolor[rgb]{0.25,0.44,0.63}{{#1}}}
    \newcommand{\StringTok}[1]{\textcolor[rgb]{0.25,0.44,0.63}{{#1}}}
    \newcommand{\CommentTok}[1]{\textcolor[rgb]{0.38,0.63,0.69}{\textit{{#1}}}}
    \newcommand{\OtherTok}[1]{\textcolor[rgb]{0.00,0.44,0.13}{{#1}}}
    \newcommand{\AlertTok}[1]{\textcolor[rgb]{1.00,0.00,0.00}{\textbf{{#1}}}}
    \newcommand{\FunctionTok}[1]{\textcolor[rgb]{0.02,0.16,0.49}{{#1}}}
    \newcommand{\RegionMarkerTok}[1]{{#1}}
    \newcommand{\ErrorTok}[1]{\textcolor[rgb]{1.00,0.00,0.00}{\textbf{{#1}}}}
    \newcommand{\NormalTok}[1]{{#1}}
    
    % Additional commands for more recent versions of Pandoc
    \newcommand{\ConstantTok}[1]{\textcolor[rgb]{0.53,0.00,0.00}{{#1}}}
    \newcommand{\SpecialCharTok}[1]{\textcolor[rgb]{0.25,0.44,0.63}{{#1}}}
    \newcommand{\VerbatimStringTok}[1]{\textcolor[rgb]{0.25,0.44,0.63}{{#1}}}
    \newcommand{\SpecialStringTok}[1]{\textcolor[rgb]{0.73,0.40,0.53}{{#1}}}
    \newcommand{\ImportTok}[1]{{#1}}
    \newcommand{\DocumentationTok}[1]{\textcolor[rgb]{0.73,0.13,0.13}{\textit{{#1}}}}
    \newcommand{\AnnotationTok}[1]{\textcolor[rgb]{0.38,0.63,0.69}{\textbf{\textit{{#1}}}}}
    \newcommand{\CommentVarTok}[1]{\textcolor[rgb]{0.38,0.63,0.69}{\textbf{\textit{{#1}}}}}
    \newcommand{\VariableTok}[1]{\textcolor[rgb]{0.10,0.09,0.49}{{#1}}}
    \newcommand{\ControlFlowTok}[1]{\textcolor[rgb]{0.00,0.44,0.13}{\textbf{{#1}}}}
    \newcommand{\OperatorTok}[1]{\textcolor[rgb]{0.40,0.40,0.40}{{#1}}}
    \newcommand{\BuiltInTok}[1]{{#1}}
    \newcommand{\ExtensionTok}[1]{{#1}}
    \newcommand{\PreprocessorTok}[1]{\textcolor[rgb]{0.74,0.48,0.00}{{#1}}}
    \newcommand{\AttributeTok}[1]{\textcolor[rgb]{0.49,0.56,0.16}{{#1}}}
    \newcommand{\InformationTok}[1]{\textcolor[rgb]{0.38,0.63,0.69}{\textbf{\textit{{#1}}}}}
    \newcommand{\WarningTok}[1]{\textcolor[rgb]{0.38,0.63,0.69}{\textbf{\textit{{#1}}}}}
    
    
    % Define a nice break command that doesn't care if a line doesn't already
    % exist.
    \def\br{\hspace*{\fill} \\* }
    % Math Jax compatibility definitions
    \def\gt{>}
    \def\lt{<}
    \let\Oldtex\TeX
    \let\Oldlatex\LaTeX
    \renewcommand{\TeX}{\textrm{\Oldtex}}
    \renewcommand{\LaTeX}{\textrm{\Oldlatex}}
    % Document parameters
    % Document title
    \title{Ejemplo}
    
    
    
    
    

    % Pygments definitions
    
\makeatletter
\def\PY@reset{\let\PY@it=\relax \let\PY@bf=\relax%
    \let\PY@ul=\relax \let\PY@tc=\relax%
    \let\PY@bc=\relax \let\PY@ff=\relax}
\def\PY@tok#1{\csname PY@tok@#1\endcsname}
\def\PY@toks#1+{\ifx\relax#1\empty\else%
    \PY@tok{#1}\expandafter\PY@toks\fi}
\def\PY@do#1{\PY@bc{\PY@tc{\PY@ul{%
    \PY@it{\PY@bf{\PY@ff{#1}}}}}}}
\def\PY#1#2{\PY@reset\PY@toks#1+\relax+\PY@do{#2}}

\expandafter\def\csname PY@tok@w\endcsname{\def\PY@tc##1{\textcolor[rgb]{0.73,0.73,0.73}{##1}}}
\expandafter\def\csname PY@tok@c\endcsname{\let\PY@it=\textit\def\PY@tc##1{\textcolor[rgb]{0.25,0.50,0.50}{##1}}}
\expandafter\def\csname PY@tok@cp\endcsname{\def\PY@tc##1{\textcolor[rgb]{0.74,0.48,0.00}{##1}}}
\expandafter\def\csname PY@tok@k\endcsname{\let\PY@bf=\textbf\def\PY@tc##1{\textcolor[rgb]{0.00,0.50,0.00}{##1}}}
\expandafter\def\csname PY@tok@kp\endcsname{\def\PY@tc##1{\textcolor[rgb]{0.00,0.50,0.00}{##1}}}
\expandafter\def\csname PY@tok@kt\endcsname{\def\PY@tc##1{\textcolor[rgb]{0.69,0.00,0.25}{##1}}}
\expandafter\def\csname PY@tok@o\endcsname{\def\PY@tc##1{\textcolor[rgb]{0.40,0.40,0.40}{##1}}}
\expandafter\def\csname PY@tok@ow\endcsname{\let\PY@bf=\textbf\def\PY@tc##1{\textcolor[rgb]{0.67,0.13,1.00}{##1}}}
\expandafter\def\csname PY@tok@nb\endcsname{\def\PY@tc##1{\textcolor[rgb]{0.00,0.50,0.00}{##1}}}
\expandafter\def\csname PY@tok@nf\endcsname{\def\PY@tc##1{\textcolor[rgb]{0.00,0.00,1.00}{##1}}}
\expandafter\def\csname PY@tok@nc\endcsname{\let\PY@bf=\textbf\def\PY@tc##1{\textcolor[rgb]{0.00,0.00,1.00}{##1}}}
\expandafter\def\csname PY@tok@nn\endcsname{\let\PY@bf=\textbf\def\PY@tc##1{\textcolor[rgb]{0.00,0.00,1.00}{##1}}}
\expandafter\def\csname PY@tok@ne\endcsname{\let\PY@bf=\textbf\def\PY@tc##1{\textcolor[rgb]{0.82,0.25,0.23}{##1}}}
\expandafter\def\csname PY@tok@nv\endcsname{\def\PY@tc##1{\textcolor[rgb]{0.10,0.09,0.49}{##1}}}
\expandafter\def\csname PY@tok@no\endcsname{\def\PY@tc##1{\textcolor[rgb]{0.53,0.00,0.00}{##1}}}
\expandafter\def\csname PY@tok@nl\endcsname{\def\PY@tc##1{\textcolor[rgb]{0.63,0.63,0.00}{##1}}}
\expandafter\def\csname PY@tok@ni\endcsname{\let\PY@bf=\textbf\def\PY@tc##1{\textcolor[rgb]{0.60,0.60,0.60}{##1}}}
\expandafter\def\csname PY@tok@na\endcsname{\def\PY@tc##1{\textcolor[rgb]{0.49,0.56,0.16}{##1}}}
\expandafter\def\csname PY@tok@nt\endcsname{\let\PY@bf=\textbf\def\PY@tc##1{\textcolor[rgb]{0.00,0.50,0.00}{##1}}}
\expandafter\def\csname PY@tok@nd\endcsname{\def\PY@tc##1{\textcolor[rgb]{0.67,0.13,1.00}{##1}}}
\expandafter\def\csname PY@tok@s\endcsname{\def\PY@tc##1{\textcolor[rgb]{0.73,0.13,0.13}{##1}}}
\expandafter\def\csname PY@tok@sd\endcsname{\let\PY@it=\textit\def\PY@tc##1{\textcolor[rgb]{0.73,0.13,0.13}{##1}}}
\expandafter\def\csname PY@tok@si\endcsname{\let\PY@bf=\textbf\def\PY@tc##1{\textcolor[rgb]{0.73,0.40,0.53}{##1}}}
\expandafter\def\csname PY@tok@se\endcsname{\let\PY@bf=\textbf\def\PY@tc##1{\textcolor[rgb]{0.73,0.40,0.13}{##1}}}
\expandafter\def\csname PY@tok@sr\endcsname{\def\PY@tc##1{\textcolor[rgb]{0.73,0.40,0.53}{##1}}}
\expandafter\def\csname PY@tok@ss\endcsname{\def\PY@tc##1{\textcolor[rgb]{0.10,0.09,0.49}{##1}}}
\expandafter\def\csname PY@tok@sx\endcsname{\def\PY@tc##1{\textcolor[rgb]{0.00,0.50,0.00}{##1}}}
\expandafter\def\csname PY@tok@m\endcsname{\def\PY@tc##1{\textcolor[rgb]{0.40,0.40,0.40}{##1}}}
\expandafter\def\csname PY@tok@gh\endcsname{\let\PY@bf=\textbf\def\PY@tc##1{\textcolor[rgb]{0.00,0.00,0.50}{##1}}}
\expandafter\def\csname PY@tok@gu\endcsname{\let\PY@bf=\textbf\def\PY@tc##1{\textcolor[rgb]{0.50,0.00,0.50}{##1}}}
\expandafter\def\csname PY@tok@gd\endcsname{\def\PY@tc##1{\textcolor[rgb]{0.63,0.00,0.00}{##1}}}
\expandafter\def\csname PY@tok@gi\endcsname{\def\PY@tc##1{\textcolor[rgb]{0.00,0.63,0.00}{##1}}}
\expandafter\def\csname PY@tok@gr\endcsname{\def\PY@tc##1{\textcolor[rgb]{1.00,0.00,0.00}{##1}}}
\expandafter\def\csname PY@tok@ge\endcsname{\let\PY@it=\textit}
\expandafter\def\csname PY@tok@gs\endcsname{\let\PY@bf=\textbf}
\expandafter\def\csname PY@tok@gp\endcsname{\let\PY@bf=\textbf\def\PY@tc##1{\textcolor[rgb]{0.00,0.00,0.50}{##1}}}
\expandafter\def\csname PY@tok@go\endcsname{\def\PY@tc##1{\textcolor[rgb]{0.53,0.53,0.53}{##1}}}
\expandafter\def\csname PY@tok@gt\endcsname{\def\PY@tc##1{\textcolor[rgb]{0.00,0.27,0.87}{##1}}}
\expandafter\def\csname PY@tok@err\endcsname{\def\PY@bc##1{\setlength{\fboxsep}{0pt}\fcolorbox[rgb]{1.00,0.00,0.00}{1,1,1}{\strut ##1}}}
\expandafter\def\csname PY@tok@kc\endcsname{\let\PY@bf=\textbf\def\PY@tc##1{\textcolor[rgb]{0.00,0.50,0.00}{##1}}}
\expandafter\def\csname PY@tok@kd\endcsname{\let\PY@bf=\textbf\def\PY@tc##1{\textcolor[rgb]{0.00,0.50,0.00}{##1}}}
\expandafter\def\csname PY@tok@kn\endcsname{\let\PY@bf=\textbf\def\PY@tc##1{\textcolor[rgb]{0.00,0.50,0.00}{##1}}}
\expandafter\def\csname PY@tok@kr\endcsname{\let\PY@bf=\textbf\def\PY@tc##1{\textcolor[rgb]{0.00,0.50,0.00}{##1}}}
\expandafter\def\csname PY@tok@bp\endcsname{\def\PY@tc##1{\textcolor[rgb]{0.00,0.50,0.00}{##1}}}
\expandafter\def\csname PY@tok@fm\endcsname{\def\PY@tc##1{\textcolor[rgb]{0.00,0.00,1.00}{##1}}}
\expandafter\def\csname PY@tok@vc\endcsname{\def\PY@tc##1{\textcolor[rgb]{0.10,0.09,0.49}{##1}}}
\expandafter\def\csname PY@tok@vg\endcsname{\def\PY@tc##1{\textcolor[rgb]{0.10,0.09,0.49}{##1}}}
\expandafter\def\csname PY@tok@vi\endcsname{\def\PY@tc##1{\textcolor[rgb]{0.10,0.09,0.49}{##1}}}
\expandafter\def\csname PY@tok@vm\endcsname{\def\PY@tc##1{\textcolor[rgb]{0.10,0.09,0.49}{##1}}}
\expandafter\def\csname PY@tok@sa\endcsname{\def\PY@tc##1{\textcolor[rgb]{0.73,0.13,0.13}{##1}}}
\expandafter\def\csname PY@tok@sb\endcsname{\def\PY@tc##1{\textcolor[rgb]{0.73,0.13,0.13}{##1}}}
\expandafter\def\csname PY@tok@sc\endcsname{\def\PY@tc##1{\textcolor[rgb]{0.73,0.13,0.13}{##1}}}
\expandafter\def\csname PY@tok@dl\endcsname{\def\PY@tc##1{\textcolor[rgb]{0.73,0.13,0.13}{##1}}}
\expandafter\def\csname PY@tok@s2\endcsname{\def\PY@tc##1{\textcolor[rgb]{0.73,0.13,0.13}{##1}}}
\expandafter\def\csname PY@tok@sh\endcsname{\def\PY@tc##1{\textcolor[rgb]{0.73,0.13,0.13}{##1}}}
\expandafter\def\csname PY@tok@s1\endcsname{\def\PY@tc##1{\textcolor[rgb]{0.73,0.13,0.13}{##1}}}
\expandafter\def\csname PY@tok@mb\endcsname{\def\PY@tc##1{\textcolor[rgb]{0.40,0.40,0.40}{##1}}}
\expandafter\def\csname PY@tok@mf\endcsname{\def\PY@tc##1{\textcolor[rgb]{0.40,0.40,0.40}{##1}}}
\expandafter\def\csname PY@tok@mh\endcsname{\def\PY@tc##1{\textcolor[rgb]{0.40,0.40,0.40}{##1}}}
\expandafter\def\csname PY@tok@mi\endcsname{\def\PY@tc##1{\textcolor[rgb]{0.40,0.40,0.40}{##1}}}
\expandafter\def\csname PY@tok@il\endcsname{\def\PY@tc##1{\textcolor[rgb]{0.40,0.40,0.40}{##1}}}
\expandafter\def\csname PY@tok@mo\endcsname{\def\PY@tc##1{\textcolor[rgb]{0.40,0.40,0.40}{##1}}}
\expandafter\def\csname PY@tok@ch\endcsname{\let\PY@it=\textit\def\PY@tc##1{\textcolor[rgb]{0.25,0.50,0.50}{##1}}}
\expandafter\def\csname PY@tok@cm\endcsname{\let\PY@it=\textit\def\PY@tc##1{\textcolor[rgb]{0.25,0.50,0.50}{##1}}}
\expandafter\def\csname PY@tok@cpf\endcsname{\let\PY@it=\textit\def\PY@tc##1{\textcolor[rgb]{0.25,0.50,0.50}{##1}}}
\expandafter\def\csname PY@tok@c1\endcsname{\let\PY@it=\textit\def\PY@tc##1{\textcolor[rgb]{0.25,0.50,0.50}{##1}}}
\expandafter\def\csname PY@tok@cs\endcsname{\let\PY@it=\textit\def\PY@tc##1{\textcolor[rgb]{0.25,0.50,0.50}{##1}}}

\def\PYZbs{\char`\\}
\def\PYZus{\char`\_}
\def\PYZob{\char`\{}
\def\PYZcb{\char`\}}
\def\PYZca{\char`\^}
\def\PYZam{\char`\&}
\def\PYZlt{\char`\<}
\def\PYZgt{\char`\>}
\def\PYZsh{\char`\#}
\def\PYZpc{\char`\%}
\def\PYZdl{\char`\$}
\def\PYZhy{\char`\-}
\def\PYZsq{\char`\'}
\def\PYZdq{\char`\"}
\def\PYZti{\char`\~}
% for compatibility with earlier versions
\def\PYZat{@}
\def\PYZlb{[}
\def\PYZrb{]}
\makeatother


    % Exact colors from NB
    \definecolor{incolor}{rgb}{0.0, 0.0, 0.5}
    \definecolor{outcolor}{rgb}{0.545, 0.0, 0.0}



    
    % Prevent overflowing lines due to hard-to-break entities
    \sloppy 
    % Setup hyperref package
    \hypersetup{
      breaklinks=true,  % so long urls are correctly broken across lines
      colorlinks=true,
      urlcolor=urlcolor,
      linkcolor=linkcolor,
      citecolor=citecolor,
      }
    % Slightly bigger margins than the latex defaults
    
    \geometry{verbose,tmargin=1in,bmargin=1in,lmargin=1in,rmargin=1in}
    
    

    \begin{document}
    
    
    \maketitle
    
    

    
    \begin{Verbatim}[commandchars=\\\{\}]
{\color{incolor}In [{\color{incolor}1}]:} \PY{k+kn}{import} \PY{n+nn}{numpy} \PY{k}{as} \PY{n+nn}{np}
        \PY{k+kn}{import} \PY{n+nn}{pandas} \PY{k}{as} \PY{n+nn}{pd}
        \PY{k+kn}{import} \PY{n+nn}{matplotlib}\PY{n+nn}{.}\PY{n+nn}{pyplot} \PY{k}{as} \PY{n+nn}{plp}
\end{Verbatim}

    \hypertarget{enunciado}{%
\subsubsection{Enunciado}\label{enunciado}}

    Obtener del sitio web del DANE (http://www.dane.gov.co/) la estimación
de la población de Colombia para el año 2017 por municipios y del sitio
de Datos Abiertos de Colombia (https://www.datos.gov.co/) los datos del
Directorio de la Red Nacional de Bibliotecas Públicas . Realizar los
siguientes procedimientos y análisis:

Organizar cada tabla en una hoja de Excel de Microsoft, una para cada
grupo de datos, leer los datos y guardarlos en dos Data Frame distintos
usando Pandas de Python.

Crear un nuevo Data Frame a partir de los dos creados en el punto
anterior con un registro por cada municipio, y con las columnas de
Nombre (Nombre del Municipio), Departamento (Departamento del
municipio), Población (Población proyectada para el año 2017) y número
de bibliotecas abiertas

Crear una tabla de frecuencia utilizando los rangos definidos de acuerdo
al número de Bibliotecas abiertas en el municipio (Cada pareja de
estudiantes debe definir el número de rangos entre 4 y 6 categorías).
Realizar un diagrama de barras usando esta tabla de frecuencias.

Clasificar los municipios de acuerdo al tamaño de su población en 5
categorías (es libre la definición de los rangos). Agregar una columna
al Data Frame creado en el punto dos con esta clasificación. Usando este
último Data Frame, calcular la media, desviación estándar, primer y
tercer cuartil del del número de Bibliotecas por cada una de las
clasificaciones de los municipios acordes al tamaño de la población.

Que conclusiones puede deducir entre la relación del del número de
Bibliotecas y el tamaño de los municipios a partir de los resultados del
punto anterior.

    \hypertarget{respuesta}{%
\subsubsection{Respuesta}\label{respuesta}}

    \hypertarget{se-obtuvo-del-sitio-web-del-dane-httpwww.dane.gov.co-la-estimaciuxf3n-de-la-poblaciuxf3n-de-colombia-para-el-auxf1o-2017-por-municipios-y-del-sitio-de-datos-abiertos-de-colombia-httpswww.datos.gov.co-los-datos-del-directorio-de-la-red-nacional-de-bibliotecas-puxfablicas.}{%
\paragraph{1. Se obtuvo del sitio web del DANE (http://www.dane.gov.co/)
la estimación de la población de Colombia para el año 2017 por
municipios y del sitio de Datos Abiertos de Colombia
(https://www.datos.gov.co/) los datos del Directorio de la Red Nacional
de Bibliotecas
Públicas.}\label{se-obtuvo-del-sitio-web-del-dane-httpwww.dane.gov.co-la-estimaciuxf3n-de-la-poblaciuxf3n-de-colombia-para-el-auxf1o-2017-por-municipios-y-del-sitio-de-datos-abiertos-de-colombia-httpswww.datos.gov.co-los-datos-del-directorio-de-la-red-nacional-de-bibliotecas-puxfablicas.}}

\hypertarget{organizar-cada-tabla-en-una-hoja-de-excel-de-microsoft-una-para-cada-grupo-de-datos-leer-los-datos-y-guardarlos-en-dos-data-frame-distintos-usando-pandas-de-python.-se-organizaron-las-tablas-en-excel-estas-dos-tablas-se-leen-y-acontinuaciuxf3n-se-muestran-los-tamauxf1os-de-los-data-frames-resultantes-y-los-primeros-10-registros-de-cada-dataframe}{%
\paragraph{\texorpdfstring{Organizar cada tabla en una hoja de Excel de
Microsoft, una para cada grupo de datos, leer los datos y guardarlos en
dos Data Frame distintos usando Pandas de Python. Se organizaron las
tablas en Excel, estas dos tablas se leen y acontinuación se muestran
los tamaños de los Data Frames resultantes y los primeros 10 registros
de cada
\emph{DataFrame}}{Organizar cada tabla en una hoja de Excel de Microsoft, una para cada grupo de datos, leer los datos y guardarlos en dos Data Frame distintos usando Pandas de Python. Se organizaron las tablas en Excel, estas dos tablas se leen y acontinuación se muestran los tamaños de los Data Frames resultantes y los primeros 10 registros de cada DataFrame}}\label{organizar-cada-tabla-en-una-hoja-de-excel-de-microsoft-una-para-cada-grupo-de-datos-leer-los-datos-y-guardarlos-en-dos-data-frame-distintos-usando-pandas-de-python.-se-organizaron-las-tablas-en-excel-estas-dos-tablas-se-leen-y-acontinuaciuxf3n-se-muestran-los-tamauxf1os-de-los-data-frames-resultantes-y-los-primeros-10-registros-de-cada-dataframe}}

    \begin{Verbatim}[commandchars=\\\{\}]
{\color{incolor}In [{\color{incolor}2}]:} \PY{n}{poblacion}\PY{o}{=}\PY{n}{pd}\PY{o}{.}\PY{n}{read\PYZus{}excel}\PY{p}{(}\PY{l+s+s2}{\PYZdq{}}\PY{l+s+s2}{poblacion2017.xlsx}\PY{l+s+s2}{\PYZdq{}}\PY{p}{,}\PY{n}{index}\PY{o}{=}\PY{k+kc}{True}\PY{p}{)}
        \PY{n}{bibliotecas}\PY{o}{=}\PY{n}{pd}\PY{o}{.}\PY{n}{read\PYZus{}excel}\PY{p}{(}\PY{l+s+s2}{\PYZdq{}}\PY{l+s+s2}{bibliotecas.xlsx}\PY{l+s+s2}{\PYZdq{}}\PY{p}{,}\PY{n}{index}\PY{o}{=}\PY{k+kc}{True}\PY{p}{)}
        \PY{n+nb}{print}\PY{p}{(}\PY{n}{poblacion}\PY{o}{.}\PY{n}{shape}\PY{p}{)}
        \PY{n+nb}{print}\PY{p}{(}\PY{n}{bibliotecas}\PY{o}{.}\PY{n}{shape}\PY{p}{)}
\end{Verbatim}

    \begin{Verbatim}[commandchars=\\\{\}]
(1122, 4)
(1455, 9)

    \end{Verbatim}

    \hypertarget{para-la-tabla-poblaciuxf3n-se-definieron-cuatro-columnas-departamento-cuxf3digo-del-municipio-nombre-del-municipio-y-tamauxf1o-de-la-poblaciuxf3n}{%
\paragraph{Para la tabla población se definieron cuatro columnas:
departamento, código del municipio, nombre del municipio y tamaño de la
población}\label{para-la-tabla-poblaciuxf3n-se-definieron-cuatro-columnas-departamento-cuxf3digo-del-municipio-nombre-del-municipio-y-tamauxf1o-de-la-poblaciuxf3n}}

    \begin{Verbatim}[commandchars=\\\{\}]
{\color{incolor}In [{\color{incolor}3}]:} \PY{n}{poblacion}\PY{o}{.}\PY{n}{head}\PY{p}{(}\PY{l+m+mi}{10}\PY{p}{)}
\end{Verbatim}

\begin{Verbatim}[commandchars=\\\{\}]
{\color{outcolor}Out[{\color{outcolor}3}]:}   Departamento  Codigo    Municipio  Poblacion
        0    Antioquia    5001     Medellín    2508452
        1    Antioquia    5002    Abejorral      19096
        2    Antioquia    5004     Abriaquí       2019
        3    Antioquia    5021   Alejandría       3393
        4    Antioquia    5030        Amagá      29980
        5    Antioquia    5031       Amalfi      22414
        6    Antioquia    5034        Andes      46621
        7    Antioquia    5036  Angelópolis       9216
        8    Antioquia    5038    Angostura      11139
        9    Antioquia    5040        Anorí      17521
\end{Verbatim}
            
    \hypertarget{en-la-tabla-para-las-bibliotecas-se-definieron-las-columnas-cuxf3digo-del-municipio-departamento-nombre-del-municipio-nombre-del-centro-poblado-tipo-de-biblioteca-nombre-estado-y-georeferenciaciuxf3n}{%
\paragraph{En la tabla para las bibliotecas se definieron las columnas
código del municipio, departamento, nombre del municipio, nombre del
centro poblado, tipo de biblioteca, nombre, estado y
georeferenciación}\label{en-la-tabla-para-las-bibliotecas-se-definieron-las-columnas-cuxf3digo-del-municipio-departamento-nombre-del-municipio-nombre-del-centro-poblado-tipo-de-biblioteca-nombre-estado-y-georeferenciaciuxf3n}}

    \begin{Verbatim}[commandchars=\\\{\}]
{\color{incolor}In [{\color{incolor}4}]:} \PY{n}{bibliotecas}\PY{o}{.}\PY{n}{head}\PY{p}{(}\PY{l+m+mi}{10}\PY{p}{)}
\end{Verbatim}

\begin{Verbatim}[commandchars=\\\{\}]
{\color{outcolor}Out[{\color{outcolor}4}]:}      Codigo Departamento       Municipio   CentroPoblado Naturaleza  \textbackslash{}
        0  91001000     AMAZONAS         LETICIA         LETICIA    ESTATAL   
        1  91530000     AMAZONAS  PUERTO ALEGRIA  PUERTO ALEGRÍA    ESTATAL   
        2  91540000     AMAZONAS   PUERTO NARIÑO   PUERTO NARIÑO    ESTATAL   
        3   5002000    ANTIOQUIA       ABEJORRAL       ABEJORRAL    ESTATAL   
        4   5004000    ANTIOQUIA        ABRIAQUI        ABRIAQUÍ    ESTATAL   
        5   5021000    ANTIOQUIA      ALEJANDRIA      ALEJANDRÍA    ESTATAL   
        6   5030000    ANTIOQUIA           AMAGA           AMAGÁ    ESTATAL   
        7   5031000    ANTIOQUIA          AMALFI          AMALFI    ESTATAL   
        8   5034000    ANTIOQUIA           ANDES           ANDES    ESTATAL   
        9   5036000    ANTIOQUIA     ANGELOPOLIS     ANGELÓPOLIS    ESTATAL   
        
                Tipo                                             Nombre   Estado  \textbackslash{}
        0  MUNICIPAL            BIBLIOTECA PÚBLICA MUNICIPAL DE LETICIA  ABIERTA   
        1      RURAL               BIBLIOTECA PÚBLICA DE PUERTO ALEGRIA  ABIERTA   
        2  MUNICIPAL            BIBLIOTECA PÚBLICA MUNICIPAL POPERAPATA  ABIERTA   
        3  MUNICIPAL  BIBLIOTECA PÚBLICA MUNICIPAL DE ABEJORRAL "JAI{\ldots}  ABIERTA   
        4  MUNICIPAL    BIBLIOTECA PÚBLICA MUNICIPAL DAVID CASTRO LÓPEZ  ABIERTA   
        5  MUNICIPAL  BIBLIOTECA PÚBLICA ALEJANDRO OSORIO CASA DE LA{\ldots}  ABIERTA   
        6  MUNICIPAL       BIBLIOTECA PÙBLICA  MUNICIPAL "EMIRO KASTOS"  ABIERTA   
        7  MUNICIPAL             BIBLIOTECA PÚBLICA MUNICIPAL DE AMALFI  ABIERTA   
        8  MUNICIPAL            BIBLIOTECA PÚBLICA GONZALO ARANGO ARIAS  ABIERTA   
        9  MUNICIPAL      BIBLIOTECA PÚBLICA MUNICIPAL ERNESTO BETANCUR  ABIERTA   
        
                        Georeferencia  
        0  (-4.2031650°,-69.9359070°)  
        1  (-1.0056436°,-74.0157723°)  
        2  (-3.7733333°,-70.3819444°)  
        3   (5.7916768°,-75.4282028°)  
        4   (6.8867393°,-75.3351273°)  
        5   (6.3770122°,-75.1403026°)  
        6   (6.0398343°,-75.7040737°)  
        7   (6.9079500°,-75.0761500°)  
        8   (5.6581043°,-75.8775382°)  
        9   (6.1096610°,-75.7118080°)  
\end{Verbatim}
            
    \hypertarget{crear-un-nuevo-data-frame-a-partir-de-los-dos-creados-en-el-punto-anterior-con-un-registro-por-cada-municipio-y-con-las-columnas-de-nombre-nombre-del-municipio-departamento-departamento-del-municipio-poblaciuxf3n-poblaciuxf3n-proyectada-para-el-auxf1o-2017-y-nuxfamero-de-bibliotecas-abiertas-20-puntos}{%
\paragraph{2. Crear un nuevo Data Frame a partir de los dos creados en
el punto anterior con un registro por cada municipio, y con las columnas
de Nombre (Nombre del Municipio), Departamento (Departamento del
municipio), Población (Población proyectada para el año 2017) y número
de bibliotecas abiertas (20
puntos)}\label{crear-un-nuevo-data-frame-a-partir-de-los-dos-creados-en-el-punto-anterior-con-un-registro-por-cada-municipio-y-con-las-columnas-de-nombre-nombre-del-municipio-departamento-departamento-del-municipio-poblaciuxf3n-poblaciuxf3n-proyectada-para-el-auxf1o-2017-y-nuxfamero-de-bibliotecas-abiertas-20-puntos}}

\hypertarget{se-utilizuxf3-el-cuxf3digo-del-municipio-para-realizar-el-conteo-de-las-bibliotecas-por-municipio-y-luego-se-agreguxf3-las-columnas-pedidas.}{%
\paragraph{Se utilizó el código del municipio para realizar el conteo de
las bibliotecas por municipio y luego se agregó las columnas
pedidas.}\label{se-utilizuxf3-el-cuxf3digo-del-municipio-para-realizar-el-conteo-de-las-bibliotecas-por-municipio-y-luego-se-agreguxf3-las-columnas-pedidas.}}

    \begin{Verbatim}[commandchars=\\\{\}]
{\color{incolor}In [{\color{incolor}5}]:} \PY{k}{def} \PY{n+nf}{extraerCodigo}\PY{p}{(}\PY{n}{codigoPrueba2}\PY{p}{)}\PY{p}{:}
            \PY{n}{texto}\PY{o}{=}\PY{l+s+s2}{\PYZdq{}}\PY{l+s+s2}{\PYZdq{}}
            \PY{k}{if}\PY{p}{(}\PY{n+nb}{len}\PY{p}{(}\PY{n+nb}{str}\PY{p}{(}\PY{n}{codigoPrueba2}\PY{p}{)}\PY{p}{)}\PY{o}{==}\PY{l+m+mi}{7}\PY{p}{)}\PY{p}{:}
                \PY{n}{texto}\PY{o}{=} \PY{n+nb}{str}\PY{p}{(}\PY{n}{codigoPrueba2}\PY{p}{)}\PY{p}{[}\PY{p}{:}\PY{l+m+mi}{4}\PY{p}{]}
            \PY{k}{else}\PY{p}{:}    
                \PY{n}{texto}\PY{o}{=} \PY{n+nb}{str}\PY{p}{(}\PY{n}{codigoPrueba2}\PY{p}{)}\PY{p}{[}\PY{p}{:}\PY{l+m+mi}{5}\PY{p}{]}
            \PY{k}{return} \PY{n}{texto}    
\end{Verbatim}

    \begin{Verbatim}[commandchars=\\\{\}]
{\color{incolor}In [{\color{incolor}6}]:} \PY{n}{bibliotecas}\PY{p}{[}\PY{l+s+s2}{\PYZdq{}}\PY{l+s+s2}{NCodigo}\PY{l+s+s2}{\PYZdq{}}\PY{p}{]}\PY{o}{=}\PY{n}{bibliotecas}\PY{p}{[}\PY{l+s+s2}{\PYZdq{}}\PY{l+s+s2}{Codigo}\PY{l+s+s2}{\PYZdq{}}\PY{p}{]}\PY{o}{.}\PY{n}{apply}\PY{p}{(}\PY{k}{lambda} \PY{n}{x}\PY{p}{:} \PY{n}{extraerCodigo}\PY{p}{(}\PY{n}{x}\PY{p}{)}\PY{p}{)}
\end{Verbatim}

    \begin{Verbatim}[commandchars=\\\{\}]
{\color{incolor}In [{\color{incolor}7}]:} \PY{n}{bibliotecasCodigo}\PY{o}{=}\PY{n}{bibliotecas}\PY{p}{[}\PY{n}{bibliotecas}\PY{p}{[}\PY{l+s+s2}{\PYZdq{}}\PY{l+s+s2}{Estado}\PY{l+s+s2}{\PYZdq{}}\PY{p}{]}\PY{o}{==}\PY{l+s+s2}{\PYZdq{}}\PY{l+s+s2}{ABIERTA}\PY{l+s+s2}{\PYZdq{}}\PY{p}{]}\PY{o}{.}\PY{n}{groupby}\PY{p}{(}\PY{l+s+s2}{\PYZdq{}}\PY{l+s+s2}{NCodigo}\PY{l+s+s2}{\PYZdq{}}\PY{p}{)}\PY{o}{.}\PY{n}{count}\PY{p}{(}\PY{p}{)}
\end{Verbatim}

    \begin{Verbatim}[commandchars=\\\{\}]
{\color{incolor}In [{\color{incolor}8}]:} \PY{n}{codigoBiblioteca}\PY{o}{=}\PY{p}{[}\PY{p}{]}
        \PY{n}{departamentoBiblioteca}\PY{o}{=}\PY{p}{[}\PY{p}{]}
        \PY{n}{municipioBiblioteca}\PY{o}{=}\PY{p}{[}\PY{p}{]}
        \PY{n}{numeroBiblioteca}\PY{o}{=}\PY{p}{[}\PY{p}{]}
        \PY{n}{poblacionBiblioteca}\PY{o}{=}\PY{p}{[}\PY{p}{]}
        \PY{k}{for} \PY{n}{indice} \PY{o+ow}{in} \PY{n}{bibliotecasCodigo}\PY{o}{.}\PY{n}{index}\PY{p}{:}    
            \PY{n}{municipio}\PY{o}{=}\PY{n}{poblacion}\PY{p}{[}\PY{n}{poblacion}\PY{o}{.}\PY{n}{Codigo}\PY{o}{==}\PY{n+nb}{int}\PY{p}{(}\PY{n}{indice}\PY{p}{)}\PY{p}{]}\PY{p}{[}\PY{p}{[}\PY{l+s+s2}{\PYZdq{}}\PY{l+s+s2}{Municipio}\PY{l+s+s2}{\PYZdq{}}\PY{p}{,}\PY{l+s+s2}{\PYZdq{}}\PY{l+s+s2}{Poblacion}\PY{l+s+s2}{\PYZdq{}}\PY{p}{,}\PY{l+s+s2}{\PYZdq{}}\PY{l+s+s2}{Departamento}\PY{l+s+s2}{\PYZdq{}}\PY{p}{]}\PY{p}{]}    
            \PY{k}{if} \PY{n}{municipio}\PY{o}{.}\PY{n}{shape}\PY{p}{[}\PY{l+m+mi}{0}\PY{p}{]}\PY{o}{\PYZgt{}}\PY{l+m+mi}{0}\PY{p}{:}        
                \PY{c+c1}{\PYZsh{}print indice,\PYZdq{} \PYZdq{},municipio.values[0][0],\PYZdq{} \PYZdq{},municipio.values[0][1]}
                \PY{n}{codigoBiblioteca}\PY{o}{.}\PY{n}{append}\PY{p}{(}\PY{n}{indice}\PY{p}{)}
                \PY{n}{municipioBiblioteca}\PY{o}{.}\PY{n}{append}\PY{p}{(}\PY{n}{municipio}\PY{o}{.}\PY{n}{values}\PY{p}{[}\PY{l+m+mi}{0}\PY{p}{]}\PY{p}{[}\PY{l+m+mi}{0}\PY{p}{]}\PY{p}{)}
                \PY{n}{poblacionBiblioteca}\PY{o}{.}\PY{n}{append}\PY{p}{(}\PY{n}{municipio}\PY{o}{.}\PY{n}{values}\PY{p}{[}\PY{l+m+mi}{0}\PY{p}{]}\PY{p}{[}\PY{l+m+mi}{1}\PY{p}{]}\PY{p}{)}
                \PY{n}{numeroBiblioteca}\PY{o}{.}\PY{n}{append}\PY{p}{(}\PY{n}{bibliotecasCodigo}\PY{o}{.}\PY{n}{loc}\PY{p}{[}\PY{n}{indice}\PY{p}{]}\PY{p}{[}\PY{l+m+mi}{0}\PY{p}{]}\PY{p}{)}
                \PY{n}{departamentoBiblioteca}\PY{o}{.}\PY{n}{append}\PY{p}{(}\PY{n}{municipio}\PY{o}{.}\PY{n}{values}\PY{p}{[}\PY{l+m+mi}{0}\PY{p}{]}\PY{p}{[}\PY{l+m+mi}{2}\PY{p}{]}\PY{p}{)}
            \PY{k}{else}\PY{p}{:}
                \PY{n+nb}{print}\PY{p}{(}\PY{l+s+s2}{\PYZdq{}}\PY{l+s+s2}{No encontrado }\PY{l+s+s2}{\PYZdq{}}\PY{p}{,}\PY{n}{indice}\PY{p}{)}    
\end{Verbatim}

    \begin{Verbatim}[commandchars=\\\{\}]
{\color{incolor}In [{\color{incolor}9}]:} \PY{n}{datos1}\PY{o}{=}\PY{n}{pd}\PY{o}{.}\PY{n}{DataFrame}\PY{p}{(}\PY{p}{\PYZob{}}\PY{l+s+s2}{\PYZdq{}}\PY{l+s+s2}{Codigo}\PY{l+s+s2}{\PYZdq{}}\PY{p}{:}\PY{n}{codigoBiblioteca}\PY{p}{,}\PY{l+s+s2}{\PYZdq{}}\PY{l+s+s2}{Municipio}\PY{l+s+s2}{\PYZdq{}}\PY{p}{:}\PY{n}{municipioBiblioteca}\PY{p}{,}
                            \PY{l+s+s2}{\PYZdq{}}\PY{l+s+s2}{Departamento}\PY{l+s+s2}{\PYZdq{}}\PY{p}{:}\PY{n}{departamentoBiblioteca}\PY{p}{,}\PY{l+s+s2}{\PYZdq{}}\PY{l+s+s2}{Poblacion}\PY{l+s+s2}{\PYZdq{}}\PY{p}{:}\PY{n}{poblacionBiblioteca}\PY{p}{,}
                             \PY{l+s+s2}{\PYZdq{}}\PY{l+s+s2}{Numero}\PY{l+s+s2}{\PYZdq{}}\PY{p}{:}\PY{n}{numeroBiblioteca}\PY{p}{\PYZcb{}}\PY{p}{)}
\end{Verbatim}

    \hypertarget{se-muestran-las-diez-primeras-filas-para-la-nueva-tabla-creada-y-el-tamauxf1o-de-la-nueva-tabla}{%
\paragraph{Se muestran las diez primeras filas para la nueva tabla
creada y el tamaño de la nueva
tabla}\label{se-muestran-las-diez-primeras-filas-para-la-nueva-tabla-creada-y-el-tamauxf1o-de-la-nueva-tabla}}

    \begin{Verbatim}[commandchars=\\\{\}]
{\color{incolor}In [{\color{incolor}10}]:} \PY{n}{datos1}\PY{o}{.}\PY{n}{head}\PY{p}{(}\PY{l+m+mi}{10}\PY{p}{)}
\end{Verbatim}

\begin{Verbatim}[commandchars=\\\{\}]
{\color{outcolor}Out[{\color{outcolor}10}]:}   Codigo          Municipio  Departamento  Poblacion  Numero
         0  11001       Bogotá, D.C.  Bogotá, D.C.    8080734      19
         1  13001          Cartagena       Bolívar    1024882      17
         2  13006               Achí       Bolívar      23851       1
         3  13030  Altos del Rosario       Bolívar      14215       1
         4  13042             Arenal       Bolívar      19743       1
         5  13052             Arjona       Bolívar      75271       1
         6  13062        Arroyohondo       Bolívar      10174       1
         7  13074   Barranco de Loba       Bolívar      18426       1
         8  13140            Calamar       Bolívar      23928       1
         9  13160         Cantagallo       Bolívar       9556       1
\end{Verbatim}
            
    \begin{Verbatim}[commandchars=\\\{\}]
{\color{incolor}In [{\color{incolor}11}]:} \PY{n}{datos1}\PY{o}{.}\PY{n}{shape}
\end{Verbatim}

\begin{Verbatim}[commandchars=\\\{\}]
{\color{outcolor}Out[{\color{outcolor}11}]:} (1094, 5)
\end{Verbatim}
            
    \hypertarget{se-comprueba-que-hay-1099-registros-diferentes-y-que-la-suma-del-nuxfamero-de-bibliotecas-es-de-1455-igual-al-nuxfamero-de-registros-originales-de-la-tabla-bibliotecas.}{%
\paragraph{Se comprueba que hay 1099 registros diferentes y que la suma
del número de bibliotecas es de 1455 igual al número de registros
originales de la tabla
bibliotecas.}\label{se-comprueba-que-hay-1099-registros-diferentes-y-que-la-suma-del-nuxfamero-de-bibliotecas-es-de-1455-igual-al-nuxfamero-de-registros-originales-de-la-tabla-bibliotecas.}}

    \hypertarget{crear-una-tabla-de-frecuencia-utilizando-los-rangos-definidos-de-acuerdo-al-nuxfamero-de-bibliotecas-abiertas-en-el-municipio-cada-pareja-de-estudiantes-debe-definir-el-nuxfamero-de-rangos-entre-4-y-6-categoruxedas.-realizar-un-diagrama-de-barras-usando-esta-tabla-de-frecuencias.}{%
\paragraph{3. Crear una tabla de frecuencia utilizando los rangos
definidos de acuerdo al número de Bibliotecas abiertas en el municipio
(Cada pareja de estudiantes debe definir el número de rangos entre 4 y 6
categorías). Realizar un diagrama de barras usando esta tabla de
frecuencias.}\label{crear-una-tabla-de-frecuencia-utilizando-los-rangos-definidos-de-acuerdo-al-nuxfamero-de-bibliotecas-abiertas-en-el-municipio-cada-pareja-de-estudiantes-debe-definir-el-nuxfamero-de-rangos-entre-4-y-6-categoruxedas.-realizar-un-diagrama-de-barras-usando-esta-tabla-de-frecuencias.}}

\hypertarget{para-realizar-la-tabla-de-frecuencias-se-utilizan-5-categoruxedas-para-el-nuxfamero-de-bibliotecas-que-se-definen-en-el-vector-bins_limits}{%
\paragraph{Para realizar la tabla de frecuencias se utilizan 5
categorías para el número de bibliotecas que se definen en el vector
bins\_limits}\label{para-realizar-la-tabla-de-frecuencias-se-utilizan-5-categoruxedas-para-el-nuxfamero-de-bibliotecas-que-se-definen-en-el-vector-bins_limits}}

    \begin{Verbatim}[commandchars=\\\{\}]
{\color{incolor}In [{\color{incolor}12}]:} \PY{n}{bins\PYZus{}limits}\PY{o}{=}\PY{p}{[}\PY{l+m+mi}{0}\PY{p}{,}\PY{l+m+mi}{2}\PY{p}{,}\PY{l+m+mi}{5}\PY{p}{,}\PY{l+m+mi}{10}\PY{p}{,}\PY{l+m+mi}{100}\PY{p}{]}
         \PY{n}{intervals\PYZus{}name}\PY{o}{=}\PY{p}{[}\PY{p}{]}
         \PY{k}{for} \PY{n}{i} \PY{o+ow}{in} \PY{n+nb}{range}\PY{p}{(}\PY{n+nb}{len}\PY{p}{(}\PY{n}{bins\PYZus{}limits}\PY{p}{)}\PY{o}{\PYZhy{}}\PY{l+m+mi}{1}\PY{p}{)}\PY{p}{:}
             \PY{n}{lim\PYZus{}inf}\PY{o}{=}\PY{n}{bins\PYZus{}limits}\PY{p}{[}\PY{n}{i}\PY{p}{]}
             \PY{n}{lim\PYZus{}sup}\PY{o}{=}\PY{n}{bins\PYZus{}limits}\PY{p}{[}\PY{n}{i}\PY{o}{+}\PY{l+m+mi}{1}\PY{p}{]}
             \PY{n}{label}\PY{o}{=}\PY{l+s+s1}{\PYZsq{}}\PY{l+s+s1}{[}\PY{l+s+s1}{\PYZsq{}}\PY{o}{+}\PY{n+nb}{str}\PY{p}{(}\PY{n}{lim\PYZus{}inf}\PY{p}{)}\PY{o}{+}\PY{l+s+s2}{\PYZdq{}}\PY{l+s+s2}{,}\PY{l+s+s2}{\PYZdq{}}\PY{o}{+}\PY{n+nb}{str}\PY{p}{(}\PY{n}{lim\PYZus{}sup}\PY{p}{)}\PY{o}{+}\PY{l+s+s2}{\PYZdq{}}\PY{l+s+s2}{)}\PY{l+s+s2}{\PYZdq{}}
             \PY{n}{intervals\PYZus{}name}\PY{o}{.}\PY{n}{append}\PY{p}{(}\PY{n}{label}\PY{p}{)}
         \PY{n}{Nbibliotecas}\PY{o}{=}\PY{n}{datos1}\PY{p}{[}\PY{l+s+s1}{\PYZsq{}}\PY{l+s+s1}{Numero}\PY{l+s+s1}{\PYZsq{}}\PY{p}{]}
         \PY{n}{x\PYZus{}freq}\PY{p}{,}\PY{n}{x\PYZus{}bins}\PY{o}{=}\PY{n}{np}\PY{o}{.}\PY{n}{histogram}\PY{p}{(}\PY{n}{Nbibliotecas}\PY{p}{,}\PY{n}{bins\PYZus{}limits}\PY{p}{)}
         \PY{n}{dataFrameP3}\PY{o}{=}\PY{n}{pd}\PY{o}{.}\PY{n}{DataFrame}\PY{o}{.}\PY{n}{from\PYZus{}items}\PY{p}{(}\PY{p}{[}\PY{p}{(}\PY{l+s+s1}{\PYZsq{}}\PY{l+s+s1}{Intervalos}\PY{l+s+s1}{\PYZsq{}}\PY{p}{,}\PY{n}{intervals\PYZus{}name}\PY{p}{)}\PY{p}{,}\PY{p}{(}\PY{l+s+s1}{\PYZsq{}}\PY{l+s+s1}{Frecuencia}\PY{l+s+s1}{\PYZsq{}}\PY{p}{,}\PY{n}{x\PYZus{}freq}\PY{p}{)}\PY{p}{]}\PY{p}{)}
\end{Verbatim}

    \begin{Verbatim}[commandchars=\\\{\}]
C:\textbackslash{}Users\textbackslash{}PORTATIL\textbackslash{}Anaconda3\textbackslash{}lib\textbackslash{}site-packages\textbackslash{}ipykernel\_launcher.py:10: FutureWarning: from\_items is deprecated. Please use DataFrame.from\_dict(dict(items), {\ldots}) instead. DataFrame.from\_dict(OrderedDict(items)) may be used to preserve the key order.
  \# Remove the CWD from sys.path while we load stuff.

    \end{Verbatim}

    \hypertarget{a-continuacion-se-muestra-la-tabla-de-frecuencias}{%
\paragraph{A continuacion se muestra la tabla de
frecuencias}\label{a-continuacion-se-muestra-la-tabla-de-frecuencias}}

    \begin{Verbatim}[commandchars=\\\{\}]
{\color{incolor}In [{\color{incolor}13}]:} \PY{n}{dataFrameP3}
\end{Verbatim}

\begin{Verbatim}[commandchars=\\\{\}]
{\color{outcolor}Out[{\color{outcolor}13}]:}   Intervalos  Frecuencia
         0      [0,2)         952
         1      [2,5)         127
         2     [5,10)           9
         3   [10,100)           6
\end{Verbatim}
            
    \hypertarget{se-comprueba-que-las-frecuencias-sumen-en-nuxfamero-de-observaciones}{%
\paragraph{Se comprueba que las frecuencias sumen en número de
observaciones}\label{se-comprueba-que-las-frecuencias-sumen-en-nuxfamero-de-observaciones}}

    \begin{Verbatim}[commandchars=\\\{\}]
{\color{incolor}In [{\color{incolor}14}]:} \PY{n}{dataFrameP3}\PY{o}{.}\PY{n}{sum}\PY{p}{(}\PY{p}{)}
\end{Verbatim}

\begin{Verbatim}[commandchars=\\\{\}]
{\color{outcolor}Out[{\color{outcolor}14}]:} Intervalos    [0,2)[2,5)[5,10)[10,100)
         Frecuencia                        1094
         dtype: object
\end{Verbatim}
            
    \hypertarget{el-diagrama-de-barras-correspondiente-es}{%
\paragraph{El diagrama de barras correspondiente
es}\label{el-diagrama-de-barras-correspondiente-es}}

    \begin{Verbatim}[commandchars=\\\{\}]
{\color{incolor}In [{\color{incolor}15}]:} \PY{o}{\PYZpc{}}\PY{k}{matplotlib} inline
         \PY{n}{dataFrameP3}\PY{o}{.}\PY{n}{plot}\PY{o}{.}\PY{n}{bar}\PY{p}{(}\PY{n}{x}\PY{o}{=}\PY{l+s+s1}{\PYZsq{}}\PY{l+s+s1}{Intervalos}\PY{l+s+s1}{\PYZsq{}}\PY{p}{,} \PY{n}{y}\PY{o}{=}\PY{l+s+s1}{\PYZsq{}}\PY{l+s+s1}{Frecuencia}\PY{l+s+s1}{\PYZsq{}}\PY{p}{,} \PY{n}{rot}\PY{o}{=}\PY{l+m+mi}{0}\PY{p}{)}
\end{Verbatim}

\begin{Verbatim}[commandchars=\\\{\}]
{\color{outcolor}Out[{\color{outcolor}15}]:} <matplotlib.axes.\_subplots.AxesSubplot at 0x153effdbac8>
\end{Verbatim}
            
    \begin{center}
    \adjustimage{max size={0.9\linewidth}{0.9\paperheight}}{output_27_1.png}
    \end{center}
    { \hspace*{\fill} \\}
    
    \hypertarget{clasificar-los-municipios-de-acuerdo-al-tamauxf1o-de-su-poblaciuxf3n-en-5-categoruxedas-es-libre-la-definiciuxf3n-de-los-rangos.-agregar-una-columna-al-data-frame-creado-en-el-punto-dos-con-esta-clasificaciuxf3n.-usando-este-uxfaltimo-data-frame-calcular-la-media-desviaciuxf3n-estuxe1ndar-primer-y-tercer-cuartil-del-del-nuxfamero-de-bibliotecas-por-cada-una-de-las-clasificaciones-de-los-municipios-acordes-al-tamauxf1o-de-la-poblaciuxf3n.}{%
\paragraph{4. Clasificar los municipios de acuerdo al tamaño de su
población en 5 categorías (es libre la definición de los rangos).
Agregar una columna al Data Frame creado en el punto dos con esta
clasificación. Usando este último Data Frame, calcular la media,
desviación estándar, primer y tercer cuartil del del número de
Bibliotecas por cada una de las clasificaciones de los municipios
acordes al tamaño de la
población.}\label{clasificar-los-municipios-de-acuerdo-al-tamauxf1o-de-su-poblaciuxf3n-en-5-categoruxedas-es-libre-la-definiciuxf3n-de-los-rangos.-agregar-una-columna-al-data-frame-creado-en-el-punto-dos-con-esta-clasificaciuxf3n.-usando-este-uxfaltimo-data-frame-calcular-la-media-desviaciuxf3n-estuxe1ndar-primer-y-tercer-cuartil-del-del-nuxfamero-de-bibliotecas-por-cada-una-de-las-clasificaciones-de-los-municipios-acordes-al-tamauxf1o-de-la-poblaciuxf3n.}}

\hypertarget{se-clasuxedfican-los-municipios-es-5-categoruxedas-de-0-5-000-5-000-10-000-10-000-100-000-100-000-1-000-000-y-de-1-000-000---9-000-000-habitantes}{%
\paragraph{Se clasífican los municipios es 5 categorías de 0-5 000, 5
000-10 000, 10 000-100 000, 100 000-1 000 000 y de 1 000 000 - 9 000 000
habitantes}\label{se-clasuxedfican-los-municipios-es-5-categoruxedas-de-0-5-000-5-000-10-000-10-000-100-000-100-000-1-000-000-y-de-1-000-000---9-000-000-habitantes}}

    \begin{Verbatim}[commandchars=\\\{\}]
{\color{incolor}In [{\color{incolor}16}]:} \PY{n}{lim1}\PY{o}{=}\PY{l+m+mi}{0}
         \PY{n}{lim2}\PY{o}{=}\PY{l+m+mi}{5001}
         \PY{n}{lim3}\PY{o}{=}\PY{l+m+mi}{10001}
         \PY{n}{lim4}\PY{o}{=}\PY{l+m+mi}{100001}
         \PY{n}{lim5}\PY{o}{=}\PY{l+m+mi}{1000001}
         \PY{n}{lim6}\PY{o}{=}\PY{l+m+mi}{10000001}
         
         \PY{n}{datos1}\PY{p}{[}\PY{l+s+s2}{\PYZdq{}}\PY{l+s+s2}{categoria}\PY{l+s+s2}{\PYZdq{}}\PY{p}{]}\PY{o}{=}\PY{n}{np}\PY{o}{.}\PY{n}{zeros}\PY{p}{(}\PY{n}{datos1}\PY{o}{.}\PY{n}{shape}\PY{p}{[}\PY{l+m+mi}{0}\PY{p}{]}\PY{p}{)}\PY{o}{.}\PY{n}{tolist}\PY{p}{(}\PY{p}{)}
         \PY{n}{datos1}\PY{o}{.}\PY{n}{loc}\PY{p}{[}\PY{p}{(}\PY{n}{datos1}\PY{o}{.}\PY{n}{Poblacion}\PY{o}{\PYZgt{}}\PY{n}{lim1}\PY{p}{)}\PY{o}{\PYZam{}}\PY{p}{(}\PY{n}{datos1}\PY{o}{.}\PY{n}{Poblacion}\PY{o}{\PYZlt{}}\PY{o}{=}\PY{n}{lim2}\PY{p}{)}\PY{p}{,}\PY{p}{[}\PY{l+s+s2}{\PYZdq{}}\PY{l+s+s2}{categoria}\PY{l+s+s2}{\PYZdq{}}\PY{p}{]}\PY{p}{]}\PY{o}{=}\PY{l+m+mi}{1}
         \PY{n}{datos1}\PY{o}{.}\PY{n}{loc}\PY{p}{[}\PY{p}{(}\PY{n}{datos1}\PY{o}{.}\PY{n}{Poblacion}\PY{o}{\PYZgt{}}\PY{n}{lim2}\PY{p}{)}\PY{o}{\PYZam{}}\PY{p}{(}\PY{n}{datos1}\PY{o}{.}\PY{n}{Poblacion}\PY{o}{\PYZlt{}}\PY{o}{=}\PY{n}{lim3}\PY{p}{)}\PY{p}{,}\PY{p}{[}\PY{l+s+s2}{\PYZdq{}}\PY{l+s+s2}{categoria}\PY{l+s+s2}{\PYZdq{}}\PY{p}{]}\PY{p}{]}\PY{o}{=}\PY{l+m+mi}{2}
         \PY{n}{datos1}\PY{o}{.}\PY{n}{loc}\PY{p}{[}\PY{p}{(}\PY{n}{datos1}\PY{o}{.}\PY{n}{Poblacion}\PY{o}{\PYZgt{}}\PY{n}{lim3}\PY{p}{)}\PY{o}{\PYZam{}}\PY{p}{(}\PY{n}{datos1}\PY{o}{.}\PY{n}{Poblacion}\PY{o}{\PYZlt{}}\PY{o}{=}\PY{n}{lim4}\PY{p}{)}\PY{p}{,}\PY{p}{[}\PY{l+s+s2}{\PYZdq{}}\PY{l+s+s2}{categoria}\PY{l+s+s2}{\PYZdq{}}\PY{p}{]}\PY{p}{]}\PY{o}{=}\PY{l+m+mi}{3}
         \PY{n}{datos1}\PY{o}{.}\PY{n}{loc}\PY{p}{[}\PY{p}{(}\PY{n}{datos1}\PY{o}{.}\PY{n}{Poblacion}\PY{o}{\PYZgt{}}\PY{n}{lim4}\PY{p}{)}\PY{o}{\PYZam{}}\PY{p}{(}\PY{n}{datos1}\PY{o}{.}\PY{n}{Poblacion}\PY{o}{\PYZlt{}}\PY{o}{=}\PY{n}{lim5}\PY{p}{)}\PY{p}{,}\PY{p}{[}\PY{l+s+s2}{\PYZdq{}}\PY{l+s+s2}{categoria}\PY{l+s+s2}{\PYZdq{}}\PY{p}{]}\PY{p}{]}\PY{o}{=}\PY{l+m+mi}{4}
         \PY{n}{datos1}\PY{o}{.}\PY{n}{loc}\PY{p}{[}\PY{p}{(}\PY{n}{datos1}\PY{o}{.}\PY{n}{Poblacion}\PY{o}{\PYZgt{}}\PY{n}{lim5}\PY{p}{)}\PY{o}{\PYZam{}}\PY{p}{(}\PY{n}{datos1}\PY{o}{.}\PY{n}{Poblacion}\PY{o}{\PYZlt{}}\PY{o}{=}\PY{n}{lim6}\PY{p}{)}\PY{p}{,}\PY{p}{[}\PY{l+s+s2}{\PYZdq{}}\PY{l+s+s2}{categoria}\PY{l+s+s2}{\PYZdq{}}\PY{p}{]}\PY{p}{]}\PY{o}{=}\PY{l+m+mi}{5}
\end{Verbatim}

    \hypertarget{se-muestran-las-primeras-filas-de-tabla-modificada}{%
\paragraph{Se muestran las primeras filas de tabla
modificada}\label{se-muestran-las-primeras-filas-de-tabla-modificada}}

    \begin{Verbatim}[commandchars=\\\{\}]
{\color{incolor}In [{\color{incolor}17}]:} \PY{n}{datos1}\PY{o}{.}\PY{n}{head}\PY{p}{(}\PY{p}{)}
\end{Verbatim}

\begin{Verbatim}[commandchars=\\\{\}]
{\color{outcolor}Out[{\color{outcolor}17}]:}   Codigo          Municipio  Departamento  Poblacion  Numero  categoria
         0  11001       Bogotá, D.C.  Bogotá, D.C.    8080734      19        5.0
         1  13001          Cartagena       Bolívar    1024882      17        5.0
         2  13006               Achí       Bolívar      23851       1        3.0
         3  13030  Altos del Rosario       Bolívar      14215       1        3.0
         4  13042             Arenal       Bolívar      19743       1        3.0
\end{Verbatim}
            
    \hypertarget{se-calcula-la-media-desviaciuxf3n-estuxe1ndar-primer-y-tercer-cuartil-por-categoria}{%
\paragraph{Se calcula la media, desviación estándar, primer y tercer
cuartil por
categoria}\label{se-calcula-la-media-desviaciuxf3n-estuxe1ndar-primer-y-tercer-cuartil-por-categoria}}

    \begin{Verbatim}[commandchars=\\\{\}]
{\color{incolor}In [{\color{incolor}18}]:} \PY{n}{datos1}\PY{o}{.}\PY{n}{groupby}\PY{p}{(}\PY{l+s+s2}{\PYZdq{}}\PY{l+s+s2}{categoria}\PY{l+s+s2}{\PYZdq{}}\PY{p}{)}\PY{p}{[}\PY{l+s+s2}{\PYZdq{}}\PY{l+s+s2}{Numero}\PY{l+s+s2}{\PYZdq{}}\PY{p}{]}\PY{o}{.}\PY{n}{describe}\PY{p}{(}\PY{p}{)}
\end{Verbatim}

\begin{Verbatim}[commandchars=\\\{\}]
{\color{outcolor}Out[{\color{outcolor}18}]:}            count       mean        std  min   25\%   50\%   75\%   max
         categoria                                                          
         1.0        166.0   1.030120   0.171436  1.0   1.0   1.0   1.0   2.0
         2.0        251.0   1.051793   0.222051  1.0   1.0   1.0   1.0   2.0
         3.0        615.0   1.186992   0.534625  1.0   1.0   1.0   1.0   5.0
         4.0         57.0   2.578947   2.764232  1.0   1.0   2.0   3.0  17.0
         5.0          5.0  26.000000  19.467922  6.0  17.0  19.0  31.0  57.0
\end{Verbatim}
            
    \hypertarget{que-conclusiones-puede-deducir-entre-la-relaciuxf3n-del-del-nuxfamero-de-bibliotecas-y-el-tamauxf1o-de-los-municipios-a-partir-de-los-resultados-del-punto-anterior.}{%
\paragraph{5. Que conclusiones puede deducir entre la relación del del
número de Bibliotecas y el tamaño de los municipios a partir de los
resultados del punto
anterior.}\label{que-conclusiones-puede-deducir-entre-la-relaciuxf3n-del-del-nuxfamero-de-bibliotecas-y-el-tamauxf1o-de-los-municipios-a-partir-de-los-resultados-del-punto-anterior.}}

Como se observa en la tabla anterior solo los municipios con más de
\(100.000\), categoría 4, habitantes poseen la mitad más de dos o más
bibliotecas, y con más de un millón de habitantes, categoría 5, la gran
mayoría posee más de 17 bibliotecas.

    \begin{Verbatim}[commandchars=\\\{\}]
{\color{incolor}In [{\color{incolor} }]:} 
\end{Verbatim}

    \begin{Verbatim}[commandchars=\\\{\}]
{\color{incolor}In [{\color{incolor} }]:} 
\end{Verbatim}

    \begin{Verbatim}[commandchars=\\\{\}]
{\color{incolor}In [{\color{incolor} }]:} 
\end{Verbatim}

    \begin{Verbatim}[commandchars=\\\{\}]
{\color{incolor}In [{\color{incolor} }]:} 
\end{Verbatim}


    % Add a bibliography block to the postdoc
    
    
    
    \end{document}
